\documentclass{article}
\usepackage[utf8]{inputenc} % Allow using of åäö, etc.
\usepackage{graphicx} % Add pictures.
\usepackage{verbatim} % Add multiline comments.

% comment about something
\author{Mikael Grön, Anssi Moisio, Visa Koski, Lassi Knuuttila}
\title{Q-learning Project Documentation - C++ ELEC-A7150}
\date{\today}

\begin{document}

\maketitle

\tableofcontents
\newpage

\section{Overview}

Main functionalities of this Q-learning implementation include:
\begin{itemize}
  \item Support for a crawler creature that has an arm with two joints. The
  crawler's purpose is to crawl to the right in a 2D-world using its arm.
  \item Many crawlers learning at the same time, in different threads.
  \item A command line user interface from which user can pause learning and
  save the Q-table.
  \item Crawlers can share their knowledge with other crawlers to speed up
  the learning process.
\end{itemize}

We could not get the graphics to work. Problems arose with
the GLUT library lacking compatibility with threading.
<visa lassi explain problems some more>

A plan-B was to visualize the Qtable with Cairo graphics library which we
configured, but we didn't have time to implement the visualization.

As a substitute for graphics, the command line interface prints out the
crawlers’ locations and speeds at regular intervals. It can be observed
from the prints that a crawler learns a maneuver by which it acquires some
top speed. After acquiring a top speed, the movement is stable (velocity is
constant) if the exploration factor is set to a low value, so that the crawler
does not make random actions.

<mikael explain the evolution and config from file>

\section{Software structure}

The main divide of the program is: learning and simulation. The
simulation and the learning parts communicate with each other mainly by the
learning part telling the simulation to do an action. The simulation will then
simulate the action, returning the new simulated change of the agent to the
learning part. The learning part evaluates the simulated change, determining
what the reward for the performed action was, updating the Q-value.

In other words, the simulation part contains the world and the body of the
agent. The learning part contains the "brain" of the agent.


\subsection{Q-learning}

The learning process takes place in the AgentLearner class.
An object of the AgentLearner
class uses a Qtable object to store and access the Q-table. A Qtable object
has a two-dimensional map structure for storing the Q-values of each
state-action pair. A map structure is used instead of, for example,
a vector structure which was the original plan, so the size of
the Q-table is possible to scale up without the access time increasing.

The Qtable class is supports dynamic creation and an object is initialized
from a vector of actors and a vector of sensors. Initializer needs
the number each actor's possible actions and the number of each
state-detecting sensor's possible states. Qtable is a map of
state-maps that contain actions for that state. The size of the Qtable
will be states * actions. If an agent has, for example, three actors
with 3, 2 and 2 possible actions, for each state there is
3 * 2 * 2 = 12 possible actions. If three state-detecting sensors can
detect each 10 different states, the total number of states is
10*10*10 = 1000. This would make a Qtable of a size 1000 * 12.

The Qtable class has functions for finding the best action and finding
the optimal Q-value for a state of the crawler. These are used in the
learning process to update the Q-values.

The AgentLearner needs to convert the state of the learning crawler to a key
for the Qtable object. The states are quantisized from the continuous states
that the crawler can have in the simulation. StateKeys are in the format
int stateKey = 141315, where each sensor's
state is represented by two digits (14, 13, 15). Sensors' states are
in the range 1...quantizationSteps. ActionKeys are in the format
int actionKey = 30302 where each sensor's
state is represented by two digits (3, 3, 2). Actors' moves are
enumerations: 0 still, 1 counterclockwise and 2 clockwise and 1 is added to them
so the key cannot be 000000.


\section{Instructions for building and using the program}

Cmake creates the Makefiles, and make uses the Makefiles to build the project.

\begin{itemize}
  \item Install required packages if they're not installed already
  \textbf{sudo apt install cmake libcairo2-dev}

  \item Move to the build folder in the q-learning-9 folder.
  \textbf{cd q-learning-9/build/}

  \item Create Makefile.
  \textbf{cmake ..}
\end{itemize}

Build targets can be listed, built together, built separately and removed:

\begin{itemize}
  \item List targets that can be built. \textbf{make help}

 \item Compile everything. \textbf{make} or \textbf{make all}

 \item Compile main. \textbf{make main}

 \item Compile tests. \textbf{make qtests}

 \item Remove compiled targets. \textbf{make clean}

 \item The executable compiled targets will be in the build folder as:
  \textbf{main} and \textbf{qtests}
\end{itemize}

\section{Testing}

Source files for testing are in the 'test' folder. Unit tests
are implemented for
the classes that pertain to the learning algorithm. more content


\section{Worklog}

\textbf{Nov 6th to 12th}

Anssi:
Configured the programming environment, i.e., linux virtual machine.
Planned the learning classes; what functions they will need and what
member data they will use. Planned the distribution of roles between
Mikael and myself, as we are the group members that implement the learning
algorithm.
15 hours

\textbf{Nov 13th to 19th}

Anssi:
Continued planning the project, made the UML picture for the plan.
3 hours

\textbf{Nov 20th to 26th}

Anssi:
Planned the Q-learning implementation with Mikael and started coding.
My responsibilities are the AgentLearning and Qtable classes.
10 hours

\textbf{Nov 27th to Dec 3rd}

Anssi:
Implemented functions in the Qtable class and in the Agent class and
re-named the Agent class as AgentLearner. Made tests for these functions.
Changed Q-table data structure from vectors to maps.
30 hours

\textbf{Dec 4th to 10th}

Anssi:
Implemented functions in the Qtable class and in the Agent class, for
converting actions and states to keys for the Q-table map and choosing
the Action. Also for saving and loading Q-table from a file.
25 hours

\textbf{Dec 11th to 17th}

Anssi:
Finalized AgentLearner class and Qtable class and implemented the
Simulation class with Visa. Tried to visualize the Q-table with
Cairo graphics library, but didn’t have time to finish. Debugged
and optimized  the learning process.
30 hours

\textbf{Dec 18th}

Anssi:
Wrote the documentation.
x hours


\end{document}
